\documentclass[10pt, letterpaper]{article}

% ---------- Fonts & layout ----------
\usepackage[margin=1in]{geometry}
\usepackage{fontspec}
\usepackage{setspace}
\usepackage{parskip}
\usepackage[dvipsnames]{xcolor}
\usepackage{tabularx}
\usepackage{hyperref}

% Set SourceSans3 as the main font
\setmainfont{SourceSans3}[
  Path = ./sourceSans/,
  Extension = .ttf,
  UprightFont = *-Regular,
  BoldFont = *-Bold,
  ItalicFont = *-Italic,
  BoldItalicFont = *-BoldItalic
]

% ---------- Colors ----------
\definecolor{Primary}{RGB}{23,43,77}
\definecolor{RuleGray}{gray}{0.25}
\definecolor{lightblue}{RGB}{10, 132, 255}

% ---------- PDF meta ----------
% TEMPLATE: Replace with candidate's name
\hypersetup {
  pdftitle={Resume - CANDIDATE\_NAME},
  pdfauthor={CANDIDATE\_NAME},
  colorlinks=true,
  linkcolor=lightblue,
  urlcolor=lightblue,
  citecolor=lightblue
}


% ---------- Section styling ----------
\makeatletter
\renewcommand{\section}{%
  \@startsection{section}{1}{\z@}{10pt}{6pt}{%
    \Large\bfseries\color{black}%
  }%
}
\renewcommand{\@seccntformat}[1]{}
\def\@sect#1#2#3#4#5#6[#7]#8{%
  \ifnum #2>\c@secnumdepth
    \let\@svsec\@empty
  \else
    \refstepcounter{#1}%
    \let\@svsec\@empty
  \fi
  \begingroup
    #6{%
      \MakeUppercase{#8}\\[-4pt]%
      \textcolor{RuleGray}{\rule{\linewidth}{0.5pt}}%
    }%
  \endgroup
  \@xsect{#5}%
}
\makeatother


% ---------- Lists ----------
\setlength{\leftmargini}{0pt}
\setlength{\itemindent}{0pt}
\setlength{\itemsep}{3pt}
\setlength{\topsep}{0pt}
\setlength{\partopsep}{0pt}
\setlength{\parsep}{0pt}

% ---------- Small helpers ----------
\newcommand{\bull}{\kern0.45em\textbullet\kern0.45em}
\newcommand{\role}[1]{\textbf{#1}}
\newcommand{\org}[1]{\textbf{#1}}
\newcommand{\place}[1]{\textit{#1}}
\newcommand{\blackLink}[2]{\href{#1}{\textcolor{black}{#2}}}
\newcommand{\blueLink}[2]{\href{#1}{\textcolor{lightblue}{#2}}}
% Template placeholder command - renders [PLACEHOLDER] without monospace font
\newcommand{\placeholder}[1]{\lbrack #1\rbrack}


% ---------- Header ----------
% TEMPLATE: Header section - replace placeholders with actual contact information
% Format: Name (large, bold) | Location | Email | Phone | LinkedIn | Website
\makeatletter
\renewcommand{\maketitle}{
  \begin{center}
    {\huge\bfseries \placeholder{CANDIDATE\_NAME}}\par\vspace{6pt}
    {\color{Primary}
      \placeholder{LOCATION} \bull
      \blackLink{mailto:\placeholder{EMAIL}}{\placeholder{EMAIL}} \bull
      \placeholder{PHONE} \bull
      \blackLink{\placeholder{LINKEDIN\_URL}}{\placeholder{LINKEDIN\_HANDLE}} \bull
      \blackLink{\placeholder{WEBSITE\_URL}}{\placeholder{WEBSITE\_DOMAIN}}
    }
  \end{center}
  \vspace{-8pt}
}
\makeatother

% ---------- Page Style ----------
\pagestyle{empty} % Add this line to remove page numbers


\begin{document}

\maketitle
\vspace{15pt}

% ---------- SUMMARY ----------
% TEMPLATE: Professional summary - 2-3 sentences highlighting key qualifications, experience level, and value proposition
\section{Summary}
\placeholder{Professional summary paragraph describing candidate's experience level, key skills, and career focus. Should be concise (2-3 sentences) and tailored to the target role.}


% ---------- EXPERIENCE ----------
% TEMPLATE: Work experience entries
% Format: \role{Job Title} \hfill Start Date -- End Date \\
%         \org{Company Name} \bull \href{URL}{Display URL} \hfill \place{City, State/Country}
%         \begin{itemize}
%           \item Achievement-focused bullet points (use \textbf{} for emphasis on key technologies/metrics)
%         \end{itemize}
%         \vspace{12pt} (between entries)
\section{Experience}
% Example entry structure - repeat this pattern for each position (most recent first):
\role{\placeholder{JOB\_TITLE}} \hfill \placeholder{START\_DATE} -- \placeholder{END\_DATE} \\
\org{\placeholder{COMPANY\_NAME}} \bull \href{\placeholder{COMPANY\_URL}}{\placeholder{COMPANY\_DOMAIN}} \hfill \place{\placeholder{LOCATION}}
\begin{itemize}
  \item \placeholder{Achievement-focused bullet point describing impact, technologies used, and quantifiable results when possible. Use \textbackslash textbf\{\} to emphasize key technologies or metrics.}
  \item \placeholder{Additional bullet points following same format - typically 3-6 bullets per position}
\end{itemize}
\vspace{12pt}
% Add more experience entries above this line, each separated by \vspace{12pt}


% ---------- PROJECTS ----------
% TEMPLATE: Personal/professional projects
% Format: \org{Project Name} \bull [Optional: Project Type] \bull \href{URL}{Display URL}
%         \begin{itemize}
%           \item Description bullets highlighting technologies, features, and impact
%         \end{itemize}
%         (blank line between projects)
\section{Projects}

% Example project structure - repeat this pattern for each project:
\org{\placeholder{PROJECT\_NAME}} \bull \placeholder{OPTIONAL\_TYPE} \bull \href{\placeholder{PROJECT\_URL}}{\placeholder{PROJECT\_DOMAIN}}
\begin{itemize}
  \item \placeholder{Project description highlighting key features, technologies used, and notable achievements or metrics}
  \item \placeholder{Additional bullets describing technical details, impact, or unique aspects}
\end{itemize}

% Add more projects above this line, each separated by a blank line

% ---------- SKILLS ----------
% TEMPLATE: Skills organized by category
% Format: \textbf{Category:} Skill1, Skill2, Skill3, ... \\
%         Continue with additional categories, each ending with \\
\section{Skills}

% Group skills into logical categories (e.g., Programming Languages, Frameworks, Tools, etc.)
% Each category should be bold, followed by comma-separated skills, ending with \\
\textbf{\placeholder{CATEGORY\_1}:} \placeholder{Skill1}, \placeholder{Skill2}, \placeholder{Skill3}, \placeholder{Skill4} \\
\textbf{\placeholder{CATEGORY\_2}:} \placeholder{Skill1}, \placeholder{Skill2}, \placeholder{Skill3} \\
\textbf{\placeholder{CATEGORY\_3}:} \placeholder{Skill1}, \placeholder{Skill2}, \placeholder{Skill3}, \placeholder{Skill4}, \placeholder{Skill5} \\
\textbf{\placeholder{CATEGORY\_4}:} \placeholder{Skill1}, \placeholder{Skill2}

% ---------- EDUCATION ----------
% TEMPLATE: Education entries
% Format: \textbf{Degree Name} \bull Institution Name \bull Location \bull Graduation Year \bull GPA: X.XX/Scale
% For multiple degrees, add additional lines separated by blank lines
\section{Education}
\textbf{\placeholder{DEGREE\_NAME}} \bull \placeholder{INSTITUTION\_NAME} \bull \placeholder{LOCATION} \bull \placeholder{GRADUATION\_YEAR} \bull GPA: \placeholder{GPA}/\placeholder{SCALE}

\end{document}
